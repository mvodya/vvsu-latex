\begin{addition}{Аннотация}
  Тема ВКР: <<Исследование морских глубин при помощи танцующих роботов>>.

  Автор работы: Студент третьего курса факультета межпланетных дел.

  Руководитель по ВКР: Ведущий научный сотрудник Института бионики.

  Цель ВКР: разработать алгоритм на базе искусственного интеллекта для координации роботов-танцоров под водой и создать на его основе комплексную систему для изучения подводного мира.

  В ходе исследования был проведен анализ проблем подводной навигации и методов синхронизации движений. Для обучения моделей был использован симулированный датасет морских течений. В процессе обучены алгоритмы координации движений, что позволило провести первоначальные тесты на эффективность синхронизации танцев. Параллельно разработана уникальная архитектура искусственного интеллекта, результаты которой были успешно интегрированы в конечную систему. Полученная система успешно прошла тестирование, подтвердив свою эффективность и надежность в экстремальных условиях.

  В ВКР содержится \printtotalpages{}, \printtotalreferences{}, \printtotalfigures{}, \printtotaltables{}.
\end{addition}