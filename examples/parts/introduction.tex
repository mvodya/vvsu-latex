\begin{introduction}
  В современном мире для обнаружения вторжений в космическое пространство широко используются различные автоматические системы, основанные на радарах. Однако у них есть значительные недостатки, включая задержку в обнаружении и трудности с выявлением вторжений на больших расстояниях.

  В случае космического вторжения стандартные радары срабатывают уже тогда, когда объект приблизился слишком близко и представляет реальную угрозу. Например, большие расстояния в космосе не способствуют своевременному обнаружению объектов, когда их еще можно было бы легко отследить. В связи с этим использование систем космического мониторинга более предпочтительно для подобных задач. Однако применение таких систем требует постоянного контроля со стороны операторов.

  Применение систем космического мониторинга исключает ошибки радаров, но ограничивает их до мониторинга человеком. Чтобы уменьшить влияние человеческого фактора и автоматизировать обнаружение вторжений, необходимо использовать системы с применением обученных искусственных интеллектов.

  Цель данной работы - исследовать и разработать алгоритм на базе машинного обучения, который позволит эффективно распознавать неизвестные объекты в космосе. Также данная работа ставит перед собой цель разработать комплексную систему обнаружения космических вторжений, в которой будет применена работа данного алгоритма.
\end{introduction}